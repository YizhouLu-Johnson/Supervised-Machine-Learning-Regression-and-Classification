\documentclass[11pt]{article}

    \usepackage[breakable]{tcolorbox}
    \usepackage{parskip} % Stop auto-indenting (to mimic markdown behaviour)
    
    \usepackage{iftex}
    \ifPDFTeX
    	\usepackage[T1]{fontenc}
    	\usepackage{mathpazo}
    \else
    	\usepackage{fontspec}
    \fi

    % Basic figure setup, for now with no caption control since it's done
    % automatically by Pandoc (which extracts ![](path) syntax from Markdown).
    \usepackage{graphicx}
    % Maintain compatibility with old templates. Remove in nbconvert 6.0
    \let\Oldincludegraphics\includegraphics
    % Ensure that by default, figures have no caption (until we provide a
    % proper Figure object with a Caption API and a way to capture that
    % in the conversion process - todo).
    \usepackage{caption}
    \DeclareCaptionFormat{nocaption}{}
    \captionsetup{format=nocaption,aboveskip=0pt,belowskip=0pt}

    \usepackage[Export]{adjustbox} % Used to constrain images to a maximum size
    \adjustboxset{max size={0.9\linewidth}{0.9\paperheight}}
    \usepackage{float}
    \floatplacement{figure}{H} % forces figures to be placed at the correct location
    \usepackage{xcolor} % Allow colors to be defined
    \usepackage{enumerate} % Needed for markdown enumerations to work
    \usepackage{geometry} % Used to adjust the document margins
    \usepackage{amsmath} % Equations
    \usepackage{amssymb} % Equations
    \usepackage{textcomp} % defines textquotesingle
    % Hack from http://tex.stackexchange.com/a/47451/13684:
    \AtBeginDocument{%
        \def\PYZsq{\textquotesingle}% Upright quotes in Pygmentized code
    }
    \usepackage{upquote} % Upright quotes for verbatim code
    \usepackage{eurosym} % defines \euro
    \usepackage[mathletters]{ucs} % Extended unicode (utf-8) support
    \usepackage{fancyvrb} % verbatim replacement that allows latex
    \usepackage{grffile} % extends the file name processing of package graphics 
                         % to support a larger range
    \makeatletter % fix for grffile with XeLaTeX
    \def\Gread@@xetex#1{%
      \IfFileExists{"\Gin@base".bb}%
      {\Gread@eps{\Gin@base.bb}}%
      {\Gread@@xetex@aux#1}%
    }
    \makeatother

    % The hyperref package gives us a pdf with properly built
    % internal navigation ('pdf bookmarks' for the table of contents,
    % internal cross-reference links, web links for URLs, etc.)
    \usepackage{hyperref}
    % The default LaTeX title has an obnoxious amount of whitespace. By default,
    % titling removes some of it. It also provides customization options.
    \usepackage{titling}
    \usepackage{longtable} % longtable support required by pandoc >1.10
    \usepackage{booktabs}  % table support for pandoc > 1.12.2
    \usepackage[inline]{enumitem} % IRkernel/repr support (it uses the enumerate* environment)
    \usepackage[normalem]{ulem} % ulem is needed to support strikethroughs (\sout)
                                % normalem makes italics be italics, not underlines
    \usepackage{mathrsfs}
    

    
    % Colors for the hyperref package
    \definecolor{urlcolor}{rgb}{0,.145,.698}
    \definecolor{linkcolor}{rgb}{.71,0.21,0.01}
    \definecolor{citecolor}{rgb}{.12,.54,.11}

    % ANSI colors
    \definecolor{ansi-black}{HTML}{3E424D}
    \definecolor{ansi-black-intense}{HTML}{282C36}
    \definecolor{ansi-red}{HTML}{E75C58}
    \definecolor{ansi-red-intense}{HTML}{B22B31}
    \definecolor{ansi-green}{HTML}{00A250}
    \definecolor{ansi-green-intense}{HTML}{007427}
    \definecolor{ansi-yellow}{HTML}{DDB62B}
    \definecolor{ansi-yellow-intense}{HTML}{B27D12}
    \definecolor{ansi-blue}{HTML}{208FFB}
    \definecolor{ansi-blue-intense}{HTML}{0065CA}
    \definecolor{ansi-magenta}{HTML}{D160C4}
    \definecolor{ansi-magenta-intense}{HTML}{A03196}
    \definecolor{ansi-cyan}{HTML}{60C6C8}
    \definecolor{ansi-cyan-intense}{HTML}{258F8F}
    \definecolor{ansi-white}{HTML}{C5C1B4}
    \definecolor{ansi-white-intense}{HTML}{A1A6B2}
    \definecolor{ansi-default-inverse-fg}{HTML}{FFFFFF}
    \definecolor{ansi-default-inverse-bg}{HTML}{000000}

    % commands and environments needed by pandoc snippets
    % extracted from the output of `pandoc -s`
    \providecommand{\tightlist}{%
      \setlength{\itemsep}{0pt}\setlength{\parskip}{0pt}}
    \DefineVerbatimEnvironment{Highlighting}{Verbatim}{commandchars=\\\{\}}
    % Add ',fontsize=\small' for more characters per line
    \newenvironment{Shaded}{}{}
    \newcommand{\KeywordTok}[1]{\textcolor[rgb]{0.00,0.44,0.13}{\textbf{{#1}}}}
    \newcommand{\DataTypeTok}[1]{\textcolor[rgb]{0.56,0.13,0.00}{{#1}}}
    \newcommand{\DecValTok}[1]{\textcolor[rgb]{0.25,0.63,0.44}{{#1}}}
    \newcommand{\BaseNTok}[1]{\textcolor[rgb]{0.25,0.63,0.44}{{#1}}}
    \newcommand{\FloatTok}[1]{\textcolor[rgb]{0.25,0.63,0.44}{{#1}}}
    \newcommand{\CharTok}[1]{\textcolor[rgb]{0.25,0.44,0.63}{{#1}}}
    \newcommand{\StringTok}[1]{\textcolor[rgb]{0.25,0.44,0.63}{{#1}}}
    \newcommand{\CommentTok}[1]{\textcolor[rgb]{0.38,0.63,0.69}{\textit{{#1}}}}
    \newcommand{\OtherTok}[1]{\textcolor[rgb]{0.00,0.44,0.13}{{#1}}}
    \newcommand{\AlertTok}[1]{\textcolor[rgb]{1.00,0.00,0.00}{\textbf{{#1}}}}
    \newcommand{\FunctionTok}[1]{\textcolor[rgb]{0.02,0.16,0.49}{{#1}}}
    \newcommand{\RegionMarkerTok}[1]{{#1}}
    \newcommand{\ErrorTok}[1]{\textcolor[rgb]{1.00,0.00,0.00}{\textbf{{#1}}}}
    \newcommand{\NormalTok}[1]{{#1}}
    
    % Additional commands for more recent versions of Pandoc
    \newcommand{\ConstantTok}[1]{\textcolor[rgb]{0.53,0.00,0.00}{{#1}}}
    \newcommand{\SpecialCharTok}[1]{\textcolor[rgb]{0.25,0.44,0.63}{{#1}}}
    \newcommand{\VerbatimStringTok}[1]{\textcolor[rgb]{0.25,0.44,0.63}{{#1}}}
    \newcommand{\SpecialStringTok}[1]{\textcolor[rgb]{0.73,0.40,0.53}{{#1}}}
    \newcommand{\ImportTok}[1]{{#1}}
    \newcommand{\DocumentationTok}[1]{\textcolor[rgb]{0.73,0.13,0.13}{\textit{{#1}}}}
    \newcommand{\AnnotationTok}[1]{\textcolor[rgb]{0.38,0.63,0.69}{\textbf{\textit{{#1}}}}}
    \newcommand{\CommentVarTok}[1]{\textcolor[rgb]{0.38,0.63,0.69}{\textbf{\textit{{#1}}}}}
    \newcommand{\VariableTok}[1]{\textcolor[rgb]{0.10,0.09,0.49}{{#1}}}
    \newcommand{\ControlFlowTok}[1]{\textcolor[rgb]{0.00,0.44,0.13}{\textbf{{#1}}}}
    \newcommand{\OperatorTok}[1]{\textcolor[rgb]{0.40,0.40,0.40}{{#1}}}
    \newcommand{\BuiltInTok}[1]{{#1}}
    \newcommand{\ExtensionTok}[1]{{#1}}
    \newcommand{\PreprocessorTok}[1]{\textcolor[rgb]{0.74,0.48,0.00}{{#1}}}
    \newcommand{\AttributeTok}[1]{\textcolor[rgb]{0.49,0.56,0.16}{{#1}}}
    \newcommand{\InformationTok}[1]{\textcolor[rgb]{0.38,0.63,0.69}{\textbf{\textit{{#1}}}}}
    \newcommand{\WarningTok}[1]{\textcolor[rgb]{0.38,0.63,0.69}{\textbf{\textit{{#1}}}}}
    
    
    % Define a nice break command that doesn't care if a line doesn't already
    % exist.
    \def\br{\hspace*{\fill} \\* }
    % Math Jax compatibility definitions
    \def\gt{>}
    \def\lt{<}
    \let\Oldtex\TeX
    \let\Oldlatex\LaTeX
    \renewcommand{\TeX}{\textrm{\Oldtex}}
    \renewcommand{\LaTeX}{\textrm{\Oldlatex}}
    % Document parameters
    % Document title
    \title{C1\_W3\_Lab09\_Regularization\_Soln}
    
    
    
    
    
% Pygments definitions
\makeatletter
\def\PY@reset{\let\PY@it=\relax \let\PY@bf=\relax%
    \let\PY@ul=\relax \let\PY@tc=\relax%
    \let\PY@bc=\relax \let\PY@ff=\relax}
\def\PY@tok#1{\csname PY@tok@#1\endcsname}
\def\PY@toks#1+{\ifx\relax#1\empty\else%
    \PY@tok{#1}\expandafter\PY@toks\fi}
\def\PY@do#1{\PY@bc{\PY@tc{\PY@ul{%
    \PY@it{\PY@bf{\PY@ff{#1}}}}}}}
\def\PY#1#2{\PY@reset\PY@toks#1+\relax+\PY@do{#2}}

\expandafter\def\csname PY@tok@w\endcsname{\def\PY@tc##1{\textcolor[rgb]{0.73,0.73,0.73}{##1}}}
\expandafter\def\csname PY@tok@c\endcsname{\let\PY@it=\textit\def\PY@tc##1{\textcolor[rgb]{0.25,0.50,0.50}{##1}}}
\expandafter\def\csname PY@tok@cp\endcsname{\def\PY@tc##1{\textcolor[rgb]{0.74,0.48,0.00}{##1}}}
\expandafter\def\csname PY@tok@k\endcsname{\let\PY@bf=\textbf\def\PY@tc##1{\textcolor[rgb]{0.00,0.50,0.00}{##1}}}
\expandafter\def\csname PY@tok@kp\endcsname{\def\PY@tc##1{\textcolor[rgb]{0.00,0.50,0.00}{##1}}}
\expandafter\def\csname PY@tok@kt\endcsname{\def\PY@tc##1{\textcolor[rgb]{0.69,0.00,0.25}{##1}}}
\expandafter\def\csname PY@tok@o\endcsname{\def\PY@tc##1{\textcolor[rgb]{0.40,0.40,0.40}{##1}}}
\expandafter\def\csname PY@tok@ow\endcsname{\let\PY@bf=\textbf\def\PY@tc##1{\textcolor[rgb]{0.67,0.13,1.00}{##1}}}
\expandafter\def\csname PY@tok@nb\endcsname{\def\PY@tc##1{\textcolor[rgb]{0.00,0.50,0.00}{##1}}}
\expandafter\def\csname PY@tok@nf\endcsname{\def\PY@tc##1{\textcolor[rgb]{0.00,0.00,1.00}{##1}}}
\expandafter\def\csname PY@tok@nc\endcsname{\let\PY@bf=\textbf\def\PY@tc##1{\textcolor[rgb]{0.00,0.00,1.00}{##1}}}
\expandafter\def\csname PY@tok@nn\endcsname{\let\PY@bf=\textbf\def\PY@tc##1{\textcolor[rgb]{0.00,0.00,1.00}{##1}}}
\expandafter\def\csname PY@tok@ne\endcsname{\let\PY@bf=\textbf\def\PY@tc##1{\textcolor[rgb]{0.82,0.25,0.23}{##1}}}
\expandafter\def\csname PY@tok@nv\endcsname{\def\PY@tc##1{\textcolor[rgb]{0.10,0.09,0.49}{##1}}}
\expandafter\def\csname PY@tok@no\endcsname{\def\PY@tc##1{\textcolor[rgb]{0.53,0.00,0.00}{##1}}}
\expandafter\def\csname PY@tok@nl\endcsname{\def\PY@tc##1{\textcolor[rgb]{0.63,0.63,0.00}{##1}}}
\expandafter\def\csname PY@tok@ni\endcsname{\let\PY@bf=\textbf\def\PY@tc##1{\textcolor[rgb]{0.60,0.60,0.60}{##1}}}
\expandafter\def\csname PY@tok@na\endcsname{\def\PY@tc##1{\textcolor[rgb]{0.49,0.56,0.16}{##1}}}
\expandafter\def\csname PY@tok@nt\endcsname{\let\PY@bf=\textbf\def\PY@tc##1{\textcolor[rgb]{0.00,0.50,0.00}{##1}}}
\expandafter\def\csname PY@tok@nd\endcsname{\def\PY@tc##1{\textcolor[rgb]{0.67,0.13,1.00}{##1}}}
\expandafter\def\csname PY@tok@s\endcsname{\def\PY@tc##1{\textcolor[rgb]{0.73,0.13,0.13}{##1}}}
\expandafter\def\csname PY@tok@sd\endcsname{\let\PY@it=\textit\def\PY@tc##1{\textcolor[rgb]{0.73,0.13,0.13}{##1}}}
\expandafter\def\csname PY@tok@si\endcsname{\let\PY@bf=\textbf\def\PY@tc##1{\textcolor[rgb]{0.73,0.40,0.53}{##1}}}
\expandafter\def\csname PY@tok@se\endcsname{\let\PY@bf=\textbf\def\PY@tc##1{\textcolor[rgb]{0.73,0.40,0.13}{##1}}}
\expandafter\def\csname PY@tok@sr\endcsname{\def\PY@tc##1{\textcolor[rgb]{0.73,0.40,0.53}{##1}}}
\expandafter\def\csname PY@tok@ss\endcsname{\def\PY@tc##1{\textcolor[rgb]{0.10,0.09,0.49}{##1}}}
\expandafter\def\csname PY@tok@sx\endcsname{\def\PY@tc##1{\textcolor[rgb]{0.00,0.50,0.00}{##1}}}
\expandafter\def\csname PY@tok@m\endcsname{\def\PY@tc##1{\textcolor[rgb]{0.40,0.40,0.40}{##1}}}
\expandafter\def\csname PY@tok@gh\endcsname{\let\PY@bf=\textbf\def\PY@tc##1{\textcolor[rgb]{0.00,0.00,0.50}{##1}}}
\expandafter\def\csname PY@tok@gu\endcsname{\let\PY@bf=\textbf\def\PY@tc##1{\textcolor[rgb]{0.50,0.00,0.50}{##1}}}
\expandafter\def\csname PY@tok@gd\endcsname{\def\PY@tc##1{\textcolor[rgb]{0.63,0.00,0.00}{##1}}}
\expandafter\def\csname PY@tok@gi\endcsname{\def\PY@tc##1{\textcolor[rgb]{0.00,0.63,0.00}{##1}}}
\expandafter\def\csname PY@tok@gr\endcsname{\def\PY@tc##1{\textcolor[rgb]{1.00,0.00,0.00}{##1}}}
\expandafter\def\csname PY@tok@ge\endcsname{\let\PY@it=\textit}
\expandafter\def\csname PY@tok@gs\endcsname{\let\PY@bf=\textbf}
\expandafter\def\csname PY@tok@gp\endcsname{\let\PY@bf=\textbf\def\PY@tc##1{\textcolor[rgb]{0.00,0.00,0.50}{##1}}}
\expandafter\def\csname PY@tok@go\endcsname{\def\PY@tc##1{\textcolor[rgb]{0.53,0.53,0.53}{##1}}}
\expandafter\def\csname PY@tok@gt\endcsname{\def\PY@tc##1{\textcolor[rgb]{0.00,0.27,0.87}{##1}}}
\expandafter\def\csname PY@tok@err\endcsname{\def\PY@bc##1{\setlength{\fboxsep}{0pt}\fcolorbox[rgb]{1.00,0.00,0.00}{1,1,1}{\strut ##1}}}
\expandafter\def\csname PY@tok@kc\endcsname{\let\PY@bf=\textbf\def\PY@tc##1{\textcolor[rgb]{0.00,0.50,0.00}{##1}}}
\expandafter\def\csname PY@tok@kd\endcsname{\let\PY@bf=\textbf\def\PY@tc##1{\textcolor[rgb]{0.00,0.50,0.00}{##1}}}
\expandafter\def\csname PY@tok@kn\endcsname{\let\PY@bf=\textbf\def\PY@tc##1{\textcolor[rgb]{0.00,0.50,0.00}{##1}}}
\expandafter\def\csname PY@tok@kr\endcsname{\let\PY@bf=\textbf\def\PY@tc##1{\textcolor[rgb]{0.00,0.50,0.00}{##1}}}
\expandafter\def\csname PY@tok@bp\endcsname{\def\PY@tc##1{\textcolor[rgb]{0.00,0.50,0.00}{##1}}}
\expandafter\def\csname PY@tok@fm\endcsname{\def\PY@tc##1{\textcolor[rgb]{0.00,0.00,1.00}{##1}}}
\expandafter\def\csname PY@tok@vc\endcsname{\def\PY@tc##1{\textcolor[rgb]{0.10,0.09,0.49}{##1}}}
\expandafter\def\csname PY@tok@vg\endcsname{\def\PY@tc##1{\textcolor[rgb]{0.10,0.09,0.49}{##1}}}
\expandafter\def\csname PY@tok@vi\endcsname{\def\PY@tc##1{\textcolor[rgb]{0.10,0.09,0.49}{##1}}}
\expandafter\def\csname PY@tok@vm\endcsname{\def\PY@tc##1{\textcolor[rgb]{0.10,0.09,0.49}{##1}}}
\expandafter\def\csname PY@tok@sa\endcsname{\def\PY@tc##1{\textcolor[rgb]{0.73,0.13,0.13}{##1}}}
\expandafter\def\csname PY@tok@sb\endcsname{\def\PY@tc##1{\textcolor[rgb]{0.73,0.13,0.13}{##1}}}
\expandafter\def\csname PY@tok@sc\endcsname{\def\PY@tc##1{\textcolor[rgb]{0.73,0.13,0.13}{##1}}}
\expandafter\def\csname PY@tok@dl\endcsname{\def\PY@tc##1{\textcolor[rgb]{0.73,0.13,0.13}{##1}}}
\expandafter\def\csname PY@tok@s2\endcsname{\def\PY@tc##1{\textcolor[rgb]{0.73,0.13,0.13}{##1}}}
\expandafter\def\csname PY@tok@sh\endcsname{\def\PY@tc##1{\textcolor[rgb]{0.73,0.13,0.13}{##1}}}
\expandafter\def\csname PY@tok@s1\endcsname{\def\PY@tc##1{\textcolor[rgb]{0.73,0.13,0.13}{##1}}}
\expandafter\def\csname PY@tok@mb\endcsname{\def\PY@tc##1{\textcolor[rgb]{0.40,0.40,0.40}{##1}}}
\expandafter\def\csname PY@tok@mf\endcsname{\def\PY@tc##1{\textcolor[rgb]{0.40,0.40,0.40}{##1}}}
\expandafter\def\csname PY@tok@mh\endcsname{\def\PY@tc##1{\textcolor[rgb]{0.40,0.40,0.40}{##1}}}
\expandafter\def\csname PY@tok@mi\endcsname{\def\PY@tc##1{\textcolor[rgb]{0.40,0.40,0.40}{##1}}}
\expandafter\def\csname PY@tok@il\endcsname{\def\PY@tc##1{\textcolor[rgb]{0.40,0.40,0.40}{##1}}}
\expandafter\def\csname PY@tok@mo\endcsname{\def\PY@tc##1{\textcolor[rgb]{0.40,0.40,0.40}{##1}}}
\expandafter\def\csname PY@tok@ch\endcsname{\let\PY@it=\textit\def\PY@tc##1{\textcolor[rgb]{0.25,0.50,0.50}{##1}}}
\expandafter\def\csname PY@tok@cm\endcsname{\let\PY@it=\textit\def\PY@tc##1{\textcolor[rgb]{0.25,0.50,0.50}{##1}}}
\expandafter\def\csname PY@tok@cpf\endcsname{\let\PY@it=\textit\def\PY@tc##1{\textcolor[rgb]{0.25,0.50,0.50}{##1}}}
\expandafter\def\csname PY@tok@c1\endcsname{\let\PY@it=\textit\def\PY@tc##1{\textcolor[rgb]{0.25,0.50,0.50}{##1}}}
\expandafter\def\csname PY@tok@cs\endcsname{\let\PY@it=\textit\def\PY@tc##1{\textcolor[rgb]{0.25,0.50,0.50}{##1}}}

\def\PYZbs{\char`\\}
\def\PYZus{\char`\_}
\def\PYZob{\char`\{}
\def\PYZcb{\char`\}}
\def\PYZca{\char`\^}
\def\PYZam{\char`\&}
\def\PYZlt{\char`\<}
\def\PYZgt{\char`\>}
\def\PYZsh{\char`\#}
\def\PYZpc{\char`\%}
\def\PYZdl{\char`\$}
\def\PYZhy{\char`\-}
\def\PYZsq{\char`\'}
\def\PYZdq{\char`\"}
\def\PYZti{\char`\~}
% for compatibility with earlier versions
\def\PYZat{@}
\def\PYZlb{[}
\def\PYZrb{]}
\makeatother


    % For linebreaks inside Verbatim environment from package fancyvrb. 
    \makeatletter
        \newbox\Wrappedcontinuationbox 
        \newbox\Wrappedvisiblespacebox 
        \newcommand*\Wrappedvisiblespace {\textcolor{red}{\textvisiblespace}} 
        \newcommand*\Wrappedcontinuationsymbol {\textcolor{red}{\llap{\tiny$\m@th\hookrightarrow$}}} 
        \newcommand*\Wrappedcontinuationindent {3ex } 
        \newcommand*\Wrappedafterbreak {\kern\Wrappedcontinuationindent\copy\Wrappedcontinuationbox} 
        % Take advantage of the already applied Pygments mark-up to insert 
        % potential linebreaks for TeX processing. 
        %        {, <, #, %, $, ' and ": go to next line. 
        %        _, }, ^, &, >, - and ~: stay at end of broken line. 
        % Use of \textquotesingle for straight quote. 
        \newcommand*\Wrappedbreaksatspecials {% 
            \def\PYGZus{\discretionary{\char`\_}{\Wrappedafterbreak}{\char`\_}}% 
            \def\PYGZob{\discretionary{}{\Wrappedafterbreak\char`\{}{\char`\{}}% 
            \def\PYGZcb{\discretionary{\char`\}}{\Wrappedafterbreak}{\char`\}}}% 
            \def\PYGZca{\discretionary{\char`\^}{\Wrappedafterbreak}{\char`\^}}% 
            \def\PYGZam{\discretionary{\char`\&}{\Wrappedafterbreak}{\char`\&}}% 
            \def\PYGZlt{\discretionary{}{\Wrappedafterbreak\char`\<}{\char`\<}}% 
            \def\PYGZgt{\discretionary{\char`\>}{\Wrappedafterbreak}{\char`\>}}% 
            \def\PYGZsh{\discretionary{}{\Wrappedafterbreak\char`\#}{\char`\#}}% 
            \def\PYGZpc{\discretionary{}{\Wrappedafterbreak\char`\%}{\char`\%}}% 
            \def\PYGZdl{\discretionary{}{\Wrappedafterbreak\char`\$}{\char`\$}}% 
            \def\PYGZhy{\discretionary{\char`\-}{\Wrappedafterbreak}{\char`\-}}% 
            \def\PYGZsq{\discretionary{}{\Wrappedafterbreak\textquotesingle}{\textquotesingle}}% 
            \def\PYGZdq{\discretionary{}{\Wrappedafterbreak\char`\"}{\char`\"}}% 
            \def\PYGZti{\discretionary{\char`\~}{\Wrappedafterbreak}{\char`\~}}% 
        } 
        % Some characters . , ; ? ! / are not pygmentized. 
        % This macro makes them "active" and they will insert potential linebreaks 
        \newcommand*\Wrappedbreaksatpunct {% 
            \lccode`\~`\.\lowercase{\def~}{\discretionary{\hbox{\char`\.}}{\Wrappedafterbreak}{\hbox{\char`\.}}}% 
            \lccode`\~`\,\lowercase{\def~}{\discretionary{\hbox{\char`\,}}{\Wrappedafterbreak}{\hbox{\char`\,}}}% 
            \lccode`\~`\;\lowercase{\def~}{\discretionary{\hbox{\char`\;}}{\Wrappedafterbreak}{\hbox{\char`\;}}}% 
            \lccode`\~`\:\lowercase{\def~}{\discretionary{\hbox{\char`\:}}{\Wrappedafterbreak}{\hbox{\char`\:}}}% 
            \lccode`\~`\?\lowercase{\def~}{\discretionary{\hbox{\char`\?}}{\Wrappedafterbreak}{\hbox{\char`\?}}}% 
            \lccode`\~`\!\lowercase{\def~}{\discretionary{\hbox{\char`\!}}{\Wrappedafterbreak}{\hbox{\char`\!}}}% 
            \lccode`\~`\/\lowercase{\def~}{\discretionary{\hbox{\char`\/}}{\Wrappedafterbreak}{\hbox{\char`\/}}}% 
            \catcode`\.\active
            \catcode`\,\active 
            \catcode`\;\active
            \catcode`\:\active
            \catcode`\?\active
            \catcode`\!\active
            \catcode`\/\active 
            \lccode`\~`\~ 	
        }
    \makeatother

    \let\OriginalVerbatim=\Verbatim
    \makeatletter
    \renewcommand{\Verbatim}[1][1]{%
        %\parskip\z@skip
        \sbox\Wrappedcontinuationbox {\Wrappedcontinuationsymbol}%
        \sbox\Wrappedvisiblespacebox {\FV@SetupFont\Wrappedvisiblespace}%
        \def\FancyVerbFormatLine ##1{\hsize\linewidth
            \vtop{\raggedright\hyphenpenalty\z@\exhyphenpenalty\z@
                \doublehyphendemerits\z@\finalhyphendemerits\z@
                \strut ##1\strut}%
        }%
        % If the linebreak is at a space, the latter will be displayed as visible
        % space at end of first line, and a continuation symbol starts next line.
        % Stretch/shrink are however usually zero for typewriter font.
        \def\FV@Space {%
            \nobreak\hskip\z@ plus\fontdimen3\font minus\fontdimen4\font
            \discretionary{\copy\Wrappedvisiblespacebox}{\Wrappedafterbreak}
            {\kern\fontdimen2\font}%
        }%
        
        % Allow breaks at special characters using \PYG... macros.
        \Wrappedbreaksatspecials
        % Breaks at punctuation characters . , ; ? ! and / need catcode=\active 	
        \OriginalVerbatim[#1,codes*=\Wrappedbreaksatpunct]%
    }
    \makeatother

    % Exact colors from NB
    \definecolor{incolor}{HTML}{303F9F}
    \definecolor{outcolor}{HTML}{D84315}
    \definecolor{cellborder}{HTML}{CFCFCF}
    \definecolor{cellbackground}{HTML}{F7F7F7}
    
    % prompt
    \makeatletter
    \newcommand{\boxspacing}{\kern\kvtcb@left@rule\kern\kvtcb@boxsep}
    \makeatother
    \newcommand{\prompt}[4]{
        \ttfamily\llap{{\color{#2}[#3]:\hspace{3pt}#4}}\vspace{-\baselineskip}
    }
    

    
    % Prevent overflowing lines due to hard-to-break entities
    \sloppy 
    % Setup hyperref package
    \hypersetup{
      breaklinks=true,  % so long urls are correctly broken across lines
      colorlinks=true,
      urlcolor=urlcolor,
      linkcolor=linkcolor,
      citecolor=citecolor,
      }
    % Slightly bigger margins than the latex defaults
    
    \geometry{verbose,tmargin=1in,bmargin=1in,lmargin=1in,rmargin=1in}
    
    

\begin{document}
    
    \maketitle
    
    

    
    \hypertarget{optional-lab---regularized-cost-and-gradient}{%
\section{Optional Lab - Regularized Cost and
Gradient}\label{optional-lab---regularized-cost-and-gradient}}

    \hypertarget{goals}{%
\subsection{Goals}\label{goals}}

In this lab, you will: - extend the previous linear and logistic cost
functions with a regularization term. - rerun the previous example of
over-fitting with a regularization term added.

    \begin{tcolorbox}[breakable, size=fbox, boxrule=1pt, pad at break*=1mm,colback=cellbackground, colframe=cellborder]
\prompt{In}{incolor}{ }{\boxspacing}
\begin{Verbatim}[commandchars=\\\{\}]
\PY{k+kn}{import} \PY{n+nn}{numpy} \PY{k}{as} \PY{n+nn}{np}
\PY{o}{\PYZpc{}}\PY{k}{matplotlib} widget
\PY{k+kn}{import} \PY{n+nn}{matplotlib}\PY{n+nn}{.}\PY{n+nn}{pyplot} \PY{k}{as} \PY{n+nn}{plt}
\PY{k+kn}{from} \PY{n+nn}{plt\PYZus{}overfit} \PY{k+kn}{import} \PY{n}{overfit\PYZus{}example}\PY{p}{,} \PY{n}{output}
\PY{k+kn}{from} \PY{n+nn}{lab\PYZus{}utils\PYZus{}common} \PY{k+kn}{import} \PY{n}{sigmoid}
\PY{n}{np}\PY{o}{.}\PY{n}{set\PYZus{}printoptions}\PY{p}{(}\PY{n}{precision}\PY{o}{=}\PY{l+m+mi}{8}\PY{p}{)}
\end{Verbatim}
\end{tcolorbox}

    \hypertarget{adding-regularization}{%
\section{Adding regularization}\label{adding-regularization}}

The slides above show the cost and gradient functions for both linear
and logistic regression. Note: - Cost - The cost functions differ
significantly between linear and logistic regression, but adding
regularization to the equations is the same. - Gradient - The gradient
functions for linear and logistic regression are very similar. They
differ only in the implementation of \(f_{wb}\).

    \hypertarget{cost-functions-with-regularization}{%
\subsection{Cost functions with
regularization}\label{cost-functions-with-regularization}}

\hypertarget{cost-function-for-regularized-linear-regression}{%
\subsubsection{Cost function for regularized linear
regression}\label{cost-function-for-regularized-linear-regression}}

The equation for the cost function regularized linear regression is:
\[J(\mathbf{w},b) = \frac{1}{2m} \sum\limits_{i = 0}^{m-1} (f_{\mathbf{w},b}(\mathbf{x}^{(i)}) - y^{(i)})^2  + \frac{\lambda}{2m}  \sum_{j=0}^{n-1} w_j^2 \tag{1}\]
where:
\[ f_{\mathbf{w},b}(\mathbf{x}^{(i)}) = \mathbf{w} \cdot \mathbf{x}^{(i)} + b  \tag{2} \]

Compare this to the cost function without regularization (which you
implemented in a previous lab), which is of the form:

\[J(\mathbf{w},b) = \frac{1}{2m} \sum\limits_{i = 0}^{m-1} (f_{\mathbf{w},b}(\mathbf{x}^{(i)}) - y^{(i)})^2 \]

The difference is the regularization term, {
\(\frac{\lambda}{2m} \sum_{j=0}^{n-1} w_j^2\) }

Including this term encourages gradient descent to minimize the size of
the parameters. Note, in this example, the parameter \(b\) is not
regularized. This is standard practice.

Below is an implementation of equations (1) and (2). Note that this uses
a \emph{standard pattern for this course}, a \texttt{for\ loop} over all
\texttt{m} examples.

    \begin{tcolorbox}[breakable, size=fbox, boxrule=1pt, pad at break*=1mm,colback=cellbackground, colframe=cellborder]
\prompt{In}{incolor}{ }{\boxspacing}
\begin{Verbatim}[commandchars=\\\{\}]
\PY{k}{def} \PY{n+nf}{compute\PYZus{}cost\PYZus{}linear\PYZus{}reg}\PY{p}{(}\PY{n}{X}\PY{p}{,} \PY{n}{y}\PY{p}{,} \PY{n}{w}\PY{p}{,} \PY{n}{b}\PY{p}{,} \PY{n}{lambda\PYZus{}} \PY{o}{=} \PY{l+m+mi}{1}\PY{p}{)}\PY{p}{:}
    \PY{l+s+sd}{\PYZdq{}\PYZdq{}\PYZdq{}}
\PY{l+s+sd}{    Computes the cost over all examples}
\PY{l+s+sd}{    Args:}
\PY{l+s+sd}{      X (ndarray (m,n): Data, m examples with n features}
\PY{l+s+sd}{      y (ndarray (m,)): target values}
\PY{l+s+sd}{      w (ndarray (n,)): model parameters  }
\PY{l+s+sd}{      b (scalar)      : model parameter}
\PY{l+s+sd}{      lambda\PYZus{} (scalar): Controls amount of regularization}
\PY{l+s+sd}{    Returns:}
\PY{l+s+sd}{      total\PYZus{}cost (scalar):  cost }
\PY{l+s+sd}{    \PYZdq{}\PYZdq{}\PYZdq{}}

    \PY{n}{m}  \PY{o}{=} \PY{n}{X}\PY{o}{.}\PY{n}{shape}\PY{p}{[}\PY{l+m+mi}{0}\PY{p}{]}
    \PY{n}{n}  \PY{o}{=} \PY{n+nb}{len}\PY{p}{(}\PY{n}{w}\PY{p}{)}
    \PY{n}{cost} \PY{o}{=} \PY{l+m+mf}{0.}
    \PY{k}{for} \PY{n}{i} \PY{o+ow}{in} \PY{n+nb}{range}\PY{p}{(}\PY{n}{m}\PY{p}{)}\PY{p}{:}
        \PY{n}{f\PYZus{}wb\PYZus{}i} \PY{o}{=} \PY{n}{np}\PY{o}{.}\PY{n}{dot}\PY{p}{(}\PY{n}{X}\PY{p}{[}\PY{n}{i}\PY{p}{]}\PY{p}{,} \PY{n}{w}\PY{p}{)} \PY{o}{+} \PY{n}{b}                                   \PY{c+c1}{\PYZsh{}(n,)(n,)=scalar, see np.dot}
        \PY{n}{cost} \PY{o}{=} \PY{n}{cost} \PY{o}{+} \PY{p}{(}\PY{n}{f\PYZus{}wb\PYZus{}i} \PY{o}{\PYZhy{}} \PY{n}{y}\PY{p}{[}\PY{n}{i}\PY{p}{]}\PY{p}{)}\PY{o}{*}\PY{o}{*}\PY{l+m+mi}{2}                               \PY{c+c1}{\PYZsh{}scalar             }
    \PY{n}{cost} \PY{o}{=} \PY{n}{cost} \PY{o}{/} \PY{p}{(}\PY{l+m+mi}{2} \PY{o}{*} \PY{n}{m}\PY{p}{)}                                              \PY{c+c1}{\PYZsh{}scalar  }
 
    \PY{n}{reg\PYZus{}cost} \PY{o}{=} \PY{l+m+mi}{0}
    \PY{k}{for} \PY{n}{j} \PY{o+ow}{in} \PY{n+nb}{range}\PY{p}{(}\PY{n}{n}\PY{p}{)}\PY{p}{:}
        \PY{n}{reg\PYZus{}cost} \PY{o}{+}\PY{o}{=} \PY{p}{(}\PY{n}{w}\PY{p}{[}\PY{n}{j}\PY{p}{]}\PY{o}{*}\PY{o}{*}\PY{l+m+mi}{2}\PY{p}{)}                                          \PY{c+c1}{\PYZsh{}scalar}
    \PY{n}{reg\PYZus{}cost} \PY{o}{=} \PY{p}{(}\PY{n}{lambda\PYZus{}}\PY{o}{/}\PY{p}{(}\PY{l+m+mi}{2}\PY{o}{*}\PY{n}{m}\PY{p}{)}\PY{p}{)} \PY{o}{*} \PY{n}{reg\PYZus{}cost}                              \PY{c+c1}{\PYZsh{}scalar}
    
    \PY{n}{total\PYZus{}cost} \PY{o}{=} \PY{n}{cost} \PY{o}{+} \PY{n}{reg\PYZus{}cost}                                       \PY{c+c1}{\PYZsh{}scalar}
    \PY{k}{return} \PY{n}{total\PYZus{}cost}                                                  \PY{c+c1}{\PYZsh{}scalar}
\end{Verbatim}
\end{tcolorbox}

    Run the cell below to see it in action.

    \begin{tcolorbox}[breakable, size=fbox, boxrule=1pt, pad at break*=1mm,colback=cellbackground, colframe=cellborder]
\prompt{In}{incolor}{ }{\boxspacing}
\begin{Verbatim}[commandchars=\\\{\}]
\PY{n}{np}\PY{o}{.}\PY{n}{random}\PY{o}{.}\PY{n}{seed}\PY{p}{(}\PY{l+m+mi}{1}\PY{p}{)}
\PY{n}{X\PYZus{}tmp} \PY{o}{=} \PY{n}{np}\PY{o}{.}\PY{n}{random}\PY{o}{.}\PY{n}{rand}\PY{p}{(}\PY{l+m+mi}{5}\PY{p}{,}\PY{l+m+mi}{6}\PY{p}{)}
\PY{n}{y\PYZus{}tmp} \PY{o}{=} \PY{n}{np}\PY{o}{.}\PY{n}{array}\PY{p}{(}\PY{p}{[}\PY{l+m+mi}{0}\PY{p}{,}\PY{l+m+mi}{1}\PY{p}{,}\PY{l+m+mi}{0}\PY{p}{,}\PY{l+m+mi}{1}\PY{p}{,}\PY{l+m+mi}{0}\PY{p}{]}\PY{p}{)}
\PY{n}{w\PYZus{}tmp} \PY{o}{=} \PY{n}{np}\PY{o}{.}\PY{n}{random}\PY{o}{.}\PY{n}{rand}\PY{p}{(}\PY{n}{X\PYZus{}tmp}\PY{o}{.}\PY{n}{shape}\PY{p}{[}\PY{l+m+mi}{1}\PY{p}{]}\PY{p}{)}\PY{o}{.}\PY{n}{reshape}\PY{p}{(}\PY{o}{\PYZhy{}}\PY{l+m+mi}{1}\PY{p}{,}\PY{p}{)}\PY{o}{\PYZhy{}}\PY{l+m+mf}{0.5}
\PY{n}{b\PYZus{}tmp} \PY{o}{=} \PY{l+m+mf}{0.5}
\PY{n}{lambda\PYZus{}tmp} \PY{o}{=} \PY{l+m+mf}{0.7}
\PY{n}{cost\PYZus{}tmp} \PY{o}{=} \PY{n}{compute\PYZus{}cost\PYZus{}linear\PYZus{}reg}\PY{p}{(}\PY{n}{X\PYZus{}tmp}\PY{p}{,} \PY{n}{y\PYZus{}tmp}\PY{p}{,} \PY{n}{w\PYZus{}tmp}\PY{p}{,} \PY{n}{b\PYZus{}tmp}\PY{p}{,} \PY{n}{lambda\PYZus{}tmp}\PY{p}{)}

\PY{n+nb}{print}\PY{p}{(}\PY{l+s+s2}{\PYZdq{}}\PY{l+s+s2}{Regularized cost:}\PY{l+s+s2}{\PYZdq{}}\PY{p}{,} \PY{n}{cost\PYZus{}tmp}\PY{p}{)}
\end{Verbatim}
\end{tcolorbox}

    \textbf{Expected Output}:

Regularized cost: 0.07917239320214275

    \hypertarget{cost-function-for-regularized-logistic-regression}{%
\subsubsection{Cost function for regularized logistic
regression}\label{cost-function-for-regularized-logistic-regression}}

For regularized \textbf{logistic} regression, the cost function is of
the form
\[J(\mathbf{w},b) = \frac{1}{m}  \sum_{i=0}^{m-1} \left[ -y^{(i)} \log\left(f_{\mathbf{w},b}\left( \mathbf{x}^{(i)} \right) \right) - \left( 1 - y^{(i)}\right) \log \left( 1 - f_{\mathbf{w},b}\left( \mathbf{x}^{(i)} \right) \right) \right] + \frac{\lambda}{2m}  \sum_{j=0}^{n-1} w_j^2 \tag{3}\]
where:
\[ f_{\mathbf{w},b}(\mathbf{x}^{(i)}) = sigmoid(\mathbf{w} \cdot \mathbf{x}^{(i)} + b)  \tag{4} \]

Compare this to the cost function without regularization (which you
implemented in a previous lab):

\[ J(\mathbf{w},b) = \frac{1}{m}\sum_{i=0}^{m-1} \left[ (-y^{(i)} \log\left(f_{\mathbf{w},b}\left( \mathbf{x}^{(i)} \right) \right) - \left( 1 - y^{(i)}\right) \log \left( 1 - f_{\mathbf{w},b}\left( \mathbf{x}^{(i)} \right) \right)\right] \]

As was the case in linear regression above, the difference is the
regularization term, which is {
\(\frac{\lambda}{2m} \sum_{j=0}^{n-1} w_j^2\) }

Including this term encourages gradient descent to minimize the size of
the parameters. Note, in this example, the parameter \(b\) is not
regularized. This is standard practice.

    \begin{tcolorbox}[breakable, size=fbox, boxrule=1pt, pad at break*=1mm,colback=cellbackground, colframe=cellborder]
\prompt{In}{incolor}{ }{\boxspacing}
\begin{Verbatim}[commandchars=\\\{\}]
\PY{k}{def} \PY{n+nf}{compute\PYZus{}cost\PYZus{}logistic\PYZus{}reg}\PY{p}{(}\PY{n}{X}\PY{p}{,} \PY{n}{y}\PY{p}{,} \PY{n}{w}\PY{p}{,} \PY{n}{b}\PY{p}{,} \PY{n}{lambda\PYZus{}} \PY{o}{=} \PY{l+m+mi}{1}\PY{p}{)}\PY{p}{:}
    \PY{l+s+sd}{\PYZdq{}\PYZdq{}\PYZdq{}}
\PY{l+s+sd}{    Computes the cost over all examples}
\PY{l+s+sd}{    Args:}
\PY{l+s+sd}{    Args:}
\PY{l+s+sd}{      X (ndarray (m,n): Data, m examples with n features}
\PY{l+s+sd}{      y (ndarray (m,)): target values}
\PY{l+s+sd}{      w (ndarray (n,)): model parameters  }
\PY{l+s+sd}{      b (scalar)      : model parameter}
\PY{l+s+sd}{      lambda\PYZus{} (scalar): Controls amount of regularization}
\PY{l+s+sd}{    Returns:}
\PY{l+s+sd}{      total\PYZus{}cost (scalar):  cost }
\PY{l+s+sd}{    \PYZdq{}\PYZdq{}\PYZdq{}}

    \PY{n}{m}\PY{p}{,}\PY{n}{n}  \PY{o}{=} \PY{n}{X}\PY{o}{.}\PY{n}{shape}
    \PY{n}{cost} \PY{o}{=} \PY{l+m+mf}{0.}
    \PY{k}{for} \PY{n}{i} \PY{o+ow}{in} \PY{n+nb}{range}\PY{p}{(}\PY{n}{m}\PY{p}{)}\PY{p}{:}
        \PY{n}{z\PYZus{}i} \PY{o}{=} \PY{n}{np}\PY{o}{.}\PY{n}{dot}\PY{p}{(}\PY{n}{X}\PY{p}{[}\PY{n}{i}\PY{p}{]}\PY{p}{,} \PY{n}{w}\PY{p}{)} \PY{o}{+} \PY{n}{b}                                      \PY{c+c1}{\PYZsh{}(n,)(n,)=scalar, see np.dot}
        \PY{n}{f\PYZus{}wb\PYZus{}i} \PY{o}{=} \PY{n}{sigmoid}\PY{p}{(}\PY{n}{z\PYZus{}i}\PY{p}{)}                                          \PY{c+c1}{\PYZsh{}scalar}
        \PY{n}{cost} \PY{o}{+}\PY{o}{=}  \PY{o}{\PYZhy{}}\PY{n}{y}\PY{p}{[}\PY{n}{i}\PY{p}{]}\PY{o}{*}\PY{n}{np}\PY{o}{.}\PY{n}{log}\PY{p}{(}\PY{n}{f\PYZus{}wb\PYZus{}i}\PY{p}{)} \PY{o}{\PYZhy{}} \PY{p}{(}\PY{l+m+mi}{1}\PY{o}{\PYZhy{}}\PY{n}{y}\PY{p}{[}\PY{n}{i}\PY{p}{]}\PY{p}{)}\PY{o}{*}\PY{n}{np}\PY{o}{.}\PY{n}{log}\PY{p}{(}\PY{l+m+mi}{1}\PY{o}{\PYZhy{}}\PY{n}{f\PYZus{}wb\PYZus{}i}\PY{p}{)}      \PY{c+c1}{\PYZsh{}scalar}
             
    \PY{n}{cost} \PY{o}{=} \PY{n}{cost}\PY{o}{/}\PY{n}{m}                                                      \PY{c+c1}{\PYZsh{}scalar}

    \PY{n}{reg\PYZus{}cost} \PY{o}{=} \PY{l+m+mi}{0}
    \PY{k}{for} \PY{n}{j} \PY{o+ow}{in} \PY{n+nb}{range}\PY{p}{(}\PY{n}{n}\PY{p}{)}\PY{p}{:}
        \PY{n}{reg\PYZus{}cost} \PY{o}{+}\PY{o}{=} \PY{p}{(}\PY{n}{w}\PY{p}{[}\PY{n}{j}\PY{p}{]}\PY{o}{*}\PY{o}{*}\PY{l+m+mi}{2}\PY{p}{)}                                          \PY{c+c1}{\PYZsh{}scalar}
    \PY{n}{reg\PYZus{}cost} \PY{o}{=} \PY{p}{(}\PY{n}{lambda\PYZus{}}\PY{o}{/}\PY{p}{(}\PY{l+m+mi}{2}\PY{o}{*}\PY{n}{m}\PY{p}{)}\PY{p}{)} \PY{o}{*} \PY{n}{reg\PYZus{}cost}                              \PY{c+c1}{\PYZsh{}scalar}
    
    \PY{n}{total\PYZus{}cost} \PY{o}{=} \PY{n}{cost} \PY{o}{+} \PY{n}{reg\PYZus{}cost}                                       \PY{c+c1}{\PYZsh{}scalar}
    \PY{k}{return} \PY{n}{total\PYZus{}cost}                                                  \PY{c+c1}{\PYZsh{}scalar}
\end{Verbatim}
\end{tcolorbox}

    Run the cell below to see it in action.

    \begin{tcolorbox}[breakable, size=fbox, boxrule=1pt, pad at break*=1mm,colback=cellbackground, colframe=cellborder]
\prompt{In}{incolor}{ }{\boxspacing}
\begin{Verbatim}[commandchars=\\\{\}]
\PY{n}{np}\PY{o}{.}\PY{n}{random}\PY{o}{.}\PY{n}{seed}\PY{p}{(}\PY{l+m+mi}{1}\PY{p}{)}
\PY{n}{X\PYZus{}tmp} \PY{o}{=} \PY{n}{np}\PY{o}{.}\PY{n}{random}\PY{o}{.}\PY{n}{rand}\PY{p}{(}\PY{l+m+mi}{5}\PY{p}{,}\PY{l+m+mi}{6}\PY{p}{)}
\PY{n}{y\PYZus{}tmp} \PY{o}{=} \PY{n}{np}\PY{o}{.}\PY{n}{array}\PY{p}{(}\PY{p}{[}\PY{l+m+mi}{0}\PY{p}{,}\PY{l+m+mi}{1}\PY{p}{,}\PY{l+m+mi}{0}\PY{p}{,}\PY{l+m+mi}{1}\PY{p}{,}\PY{l+m+mi}{0}\PY{p}{]}\PY{p}{)}
\PY{n}{w\PYZus{}tmp} \PY{o}{=} \PY{n}{np}\PY{o}{.}\PY{n}{random}\PY{o}{.}\PY{n}{rand}\PY{p}{(}\PY{n}{X\PYZus{}tmp}\PY{o}{.}\PY{n}{shape}\PY{p}{[}\PY{l+m+mi}{1}\PY{p}{]}\PY{p}{)}\PY{o}{.}\PY{n}{reshape}\PY{p}{(}\PY{o}{\PYZhy{}}\PY{l+m+mi}{1}\PY{p}{,}\PY{p}{)}\PY{o}{\PYZhy{}}\PY{l+m+mf}{0.5}
\PY{n}{b\PYZus{}tmp} \PY{o}{=} \PY{l+m+mf}{0.5}
\PY{n}{lambda\PYZus{}tmp} \PY{o}{=} \PY{l+m+mf}{0.7}
\PY{n}{cost\PYZus{}tmp} \PY{o}{=} \PY{n}{compute\PYZus{}cost\PYZus{}logistic\PYZus{}reg}\PY{p}{(}\PY{n}{X\PYZus{}tmp}\PY{p}{,} \PY{n}{y\PYZus{}tmp}\PY{p}{,} \PY{n}{w\PYZus{}tmp}\PY{p}{,} \PY{n}{b\PYZus{}tmp}\PY{p}{,} \PY{n}{lambda\PYZus{}tmp}\PY{p}{)}

\PY{n+nb}{print}\PY{p}{(}\PY{l+s+s2}{\PYZdq{}}\PY{l+s+s2}{Regularized cost:}\PY{l+s+s2}{\PYZdq{}}\PY{p}{,} \PY{n}{cost\PYZus{}tmp}\PY{p}{)}
\end{Verbatim}
\end{tcolorbox}

    \textbf{Expected Output}:

Regularized cost: 0.6850849138741673

    \hypertarget{gradient-descent-with-regularization}{%
\subsection{Gradient descent with
regularization}\label{gradient-descent-with-regularization}}

The basic algorithm for running gradient descent does not change with
regularization, it is: \[\begin{align*}
&\text{repeat until convergence:} \; \lbrace \\
&  \; \; \;w_j = w_j -  \alpha \frac{\partial J(\mathbf{w},b)}{\partial w_j} \tag{1}  \; & \text{for j := 0..n-1} \\ 
&  \; \; \;  \; \;b = b -  \alpha \frac{\partial J(\mathbf{w},b)}{\partial b} \\
&\rbrace
\end{align*}\] Where each iteration performs simultaneous updates on
\(w_j\) for all \(j\).

What changes with regularization is computing the gradients.

    \hypertarget{computing-the-gradient-with-regularization-both-linearlogistic}{%
\subsubsection{Computing the Gradient with regularization (both
linear/logistic)}\label{computing-the-gradient-with-regularization-both-linearlogistic}}

The gradient calculation for both linear and logistic regression are
nearly identical, differing only in computation of \(f_{\mathbf{w}b}\).
\[\begin{align*}
\frac{\partial J(\mathbf{w},b)}{\partial w_j}  &= \frac{1}{m} \sum\limits_{i = 0}^{m-1} (f_{\mathbf{w},b}(\mathbf{x}^{(i)}) - y^{(i)})x_{j}^{(i)}  +  \frac{\lambda}{m} w_j \tag{2} \\
\frac{\partial J(\mathbf{w},b)}{\partial b}  &= \frac{1}{m} \sum\limits_{i = 0}^{m-1} (f_{\mathbf{w},b}(\mathbf{x}^{(i)}) - y^{(i)}) \tag{3} 
\end{align*}\]

\begin{itemize}
\item
  m is the number of training examples in the data set\\
\item
  \(f_{\mathbf{w},b}(x^{(i)})\) is the model's prediction, while
  \(y^{(i)}\) is the target
\item
  For a { \textbf{linear} } regression model\\
  \(f_{\mathbf{w},b}(x) = \mathbf{w} \cdot \mathbf{x} + b\)\\
\item
  For a { \textbf{logistic} } regression model\\
  \(z = \mathbf{w} \cdot \mathbf{x} + b\)\\
  \(f_{\mathbf{w},b}(x) = g(z)\)\\
  where \(g(z)\) is the sigmoid function:\\
  \(g(z) = \frac{1}{1+e^{-z}}\)
\end{itemize}

The term which adds regularization is the {\$\frac{\lambda}{m} w\_j \$}.

    \hypertarget{gradient-function-for-regularized-linear-regression}{%
\subsubsection{Gradient function for regularized linear
regression}\label{gradient-function-for-regularized-linear-regression}}

    \begin{tcolorbox}[breakable, size=fbox, boxrule=1pt, pad at break*=1mm,colback=cellbackground, colframe=cellborder]
\prompt{In}{incolor}{ }{\boxspacing}
\begin{Verbatim}[commandchars=\\\{\}]
\PY{k}{def} \PY{n+nf}{compute\PYZus{}gradient\PYZus{}linear\PYZus{}reg}\PY{p}{(}\PY{n}{X}\PY{p}{,} \PY{n}{y}\PY{p}{,} \PY{n}{w}\PY{p}{,} \PY{n}{b}\PY{p}{,} \PY{n}{lambda\PYZus{}}\PY{p}{)}\PY{p}{:} 
    \PY{l+s+sd}{\PYZdq{}\PYZdq{}\PYZdq{}}
\PY{l+s+sd}{    Computes the gradient for linear regression }
\PY{l+s+sd}{    Args:}
\PY{l+s+sd}{      X (ndarray (m,n): Data, m examples with n features}
\PY{l+s+sd}{      y (ndarray (m,)): target values}
\PY{l+s+sd}{      w (ndarray (n,)): model parameters  }
\PY{l+s+sd}{      b (scalar)      : model parameter}
\PY{l+s+sd}{      lambda\PYZus{} (scalar): Controls amount of regularization}
\PY{l+s+sd}{      }
\PY{l+s+sd}{    Returns:}
\PY{l+s+sd}{      dj\PYZus{}dw (ndarray (n,)): The gradient of the cost w.r.t. the parameters w. }
\PY{l+s+sd}{      dj\PYZus{}db (scalar):       The gradient of the cost w.r.t. the parameter b. }
\PY{l+s+sd}{    \PYZdq{}\PYZdq{}\PYZdq{}}
    \PY{n}{m}\PY{p}{,}\PY{n}{n} \PY{o}{=} \PY{n}{X}\PY{o}{.}\PY{n}{shape}           \PY{c+c1}{\PYZsh{}(number of examples, number of features)}
    \PY{n}{dj\PYZus{}dw} \PY{o}{=} \PY{n}{np}\PY{o}{.}\PY{n}{zeros}\PY{p}{(}\PY{p}{(}\PY{n}{n}\PY{p}{,}\PY{p}{)}\PY{p}{)}
    \PY{n}{dj\PYZus{}db} \PY{o}{=} \PY{l+m+mf}{0.}

    \PY{k}{for} \PY{n}{i} \PY{o+ow}{in} \PY{n+nb}{range}\PY{p}{(}\PY{n}{m}\PY{p}{)}\PY{p}{:}                             
        \PY{n}{err} \PY{o}{=} \PY{p}{(}\PY{n}{np}\PY{o}{.}\PY{n}{dot}\PY{p}{(}\PY{n}{X}\PY{p}{[}\PY{n}{i}\PY{p}{]}\PY{p}{,} \PY{n}{w}\PY{p}{)} \PY{o}{+} \PY{n}{b}\PY{p}{)} \PY{o}{\PYZhy{}} \PY{n}{y}\PY{p}{[}\PY{n}{i}\PY{p}{]}                 
        \PY{k}{for} \PY{n}{j} \PY{o+ow}{in} \PY{n+nb}{range}\PY{p}{(}\PY{n}{n}\PY{p}{)}\PY{p}{:}                         
            \PY{n}{dj\PYZus{}dw}\PY{p}{[}\PY{n}{j}\PY{p}{]} \PY{o}{=} \PY{n}{dj\PYZus{}dw}\PY{p}{[}\PY{n}{j}\PY{p}{]} \PY{o}{+} \PY{n}{err} \PY{o}{*} \PY{n}{X}\PY{p}{[}\PY{n}{i}\PY{p}{,} \PY{n}{j}\PY{p}{]}               
        \PY{n}{dj\PYZus{}db} \PY{o}{=} \PY{n}{dj\PYZus{}db} \PY{o}{+} \PY{n}{err}                        
    \PY{n}{dj\PYZus{}dw} \PY{o}{=} \PY{n}{dj\PYZus{}dw} \PY{o}{/} \PY{n}{m}                                
    \PY{n}{dj\PYZus{}db} \PY{o}{=} \PY{n}{dj\PYZus{}db} \PY{o}{/} \PY{n}{m}   
    
    \PY{k}{for} \PY{n}{j} \PY{o+ow}{in} \PY{n+nb}{range}\PY{p}{(}\PY{n}{n}\PY{p}{)}\PY{p}{:}
        \PY{n}{dj\PYZus{}dw}\PY{p}{[}\PY{n}{j}\PY{p}{]} \PY{o}{=} \PY{n}{dj\PYZus{}dw}\PY{p}{[}\PY{n}{j}\PY{p}{]} \PY{o}{+} \PY{p}{(}\PY{n}{lambda\PYZus{}}\PY{o}{/}\PY{n}{m}\PY{p}{)} \PY{o}{*} \PY{n}{w}\PY{p}{[}\PY{n}{j}\PY{p}{]}

    \PY{k}{return} \PY{n}{dj\PYZus{}db}\PY{p}{,} \PY{n}{dj\PYZus{}dw}
\end{Verbatim}
\end{tcolorbox}

    Run the cell below to see it in action.

    \begin{tcolorbox}[breakable, size=fbox, boxrule=1pt, pad at break*=1mm,colback=cellbackground, colframe=cellborder]
\prompt{In}{incolor}{ }{\boxspacing}
\begin{Verbatim}[commandchars=\\\{\}]
\PY{n}{np}\PY{o}{.}\PY{n}{random}\PY{o}{.}\PY{n}{seed}\PY{p}{(}\PY{l+m+mi}{1}\PY{p}{)}
\PY{n}{X\PYZus{}tmp} \PY{o}{=} \PY{n}{np}\PY{o}{.}\PY{n}{random}\PY{o}{.}\PY{n}{rand}\PY{p}{(}\PY{l+m+mi}{5}\PY{p}{,}\PY{l+m+mi}{3}\PY{p}{)}
\PY{n}{y\PYZus{}tmp} \PY{o}{=} \PY{n}{np}\PY{o}{.}\PY{n}{array}\PY{p}{(}\PY{p}{[}\PY{l+m+mi}{0}\PY{p}{,}\PY{l+m+mi}{1}\PY{p}{,}\PY{l+m+mi}{0}\PY{p}{,}\PY{l+m+mi}{1}\PY{p}{,}\PY{l+m+mi}{0}\PY{p}{]}\PY{p}{)}
\PY{n}{w\PYZus{}tmp} \PY{o}{=} \PY{n}{np}\PY{o}{.}\PY{n}{random}\PY{o}{.}\PY{n}{rand}\PY{p}{(}\PY{n}{X\PYZus{}tmp}\PY{o}{.}\PY{n}{shape}\PY{p}{[}\PY{l+m+mi}{1}\PY{p}{]}\PY{p}{)}
\PY{n}{b\PYZus{}tmp} \PY{o}{=} \PY{l+m+mf}{0.5}
\PY{n}{lambda\PYZus{}tmp} \PY{o}{=} \PY{l+m+mf}{0.7}
\PY{n}{dj\PYZus{}db\PYZus{}tmp}\PY{p}{,} \PY{n}{dj\PYZus{}dw\PYZus{}tmp} \PY{o}{=}  \PY{n}{compute\PYZus{}gradient\PYZus{}linear\PYZus{}reg}\PY{p}{(}\PY{n}{X\PYZus{}tmp}\PY{p}{,} \PY{n}{y\PYZus{}tmp}\PY{p}{,} \PY{n}{w\PYZus{}tmp}\PY{p}{,} \PY{n}{b\PYZus{}tmp}\PY{p}{,} \PY{n}{lambda\PYZus{}tmp}\PY{p}{)}

\PY{n+nb}{print}\PY{p}{(}\PY{l+s+sa}{f}\PY{l+s+s2}{\PYZdq{}}\PY{l+s+s2}{dj\PYZus{}db: }\PY{l+s+si}{\PYZob{}}\PY{n}{dj\PYZus{}db\PYZus{}tmp}\PY{l+s+si}{\PYZcb{}}\PY{l+s+s2}{\PYZdq{}}\PY{p}{,} \PY{p}{)}
\PY{n+nb}{print}\PY{p}{(}\PY{l+s+sa}{f}\PY{l+s+s2}{\PYZdq{}}\PY{l+s+s2}{Regularized dj\PYZus{}dw:}\PY{l+s+se}{\PYZbs{}n}\PY{l+s+s2}{ }\PY{l+s+si}{\PYZob{}}\PY{n}{dj\PYZus{}dw\PYZus{}tmp}\PY{o}{.}\PY{n}{tolist}\PY{p}{(}\PY{p}{)}\PY{l+s+si}{\PYZcb{}}\PY{l+s+s2}{\PYZdq{}}\PY{p}{,} \PY{p}{)}
\end{Verbatim}
\end{tcolorbox}

    \textbf{Expected Output}

\begin{verbatim}
dj_db: 0.6648774569425726
Regularized dj_dw:
 [0.29653214748822276, 0.4911679625918033, 0.21645877535865857]
\end{verbatim}

    \hypertarget{gradient-function-for-regularized-logistic-regression}{%
\subsubsection{Gradient function for regularized logistic
regression}\label{gradient-function-for-regularized-logistic-regression}}

    \begin{tcolorbox}[breakable, size=fbox, boxrule=1pt, pad at break*=1mm,colback=cellbackground, colframe=cellborder]
\prompt{In}{incolor}{ }{\boxspacing}
\begin{Verbatim}[commandchars=\\\{\}]
\PY{k}{def} \PY{n+nf}{compute\PYZus{}gradient\PYZus{}logistic\PYZus{}reg}\PY{p}{(}\PY{n}{X}\PY{p}{,} \PY{n}{y}\PY{p}{,} \PY{n}{w}\PY{p}{,} \PY{n}{b}\PY{p}{,} \PY{n}{lambda\PYZus{}}\PY{p}{)}\PY{p}{:} 
    \PY{l+s+sd}{\PYZdq{}\PYZdq{}\PYZdq{}}
\PY{l+s+sd}{    Computes the gradient for linear regression }
\PY{l+s+sd}{ }
\PY{l+s+sd}{    Args:}
\PY{l+s+sd}{      X (ndarray (m,n): Data, m examples with n features}
\PY{l+s+sd}{      y (ndarray (m,)): target values}
\PY{l+s+sd}{      w (ndarray (n,)): model parameters  }
\PY{l+s+sd}{      b (scalar)      : model parameter}
\PY{l+s+sd}{      lambda\PYZus{} (scalar): Controls amount of regularization}
\PY{l+s+sd}{    Returns}
\PY{l+s+sd}{      dj\PYZus{}dw (ndarray Shape (n,)): The gradient of the cost w.r.t. the parameters w. }
\PY{l+s+sd}{      dj\PYZus{}db (scalar)            : The gradient of the cost w.r.t. the parameter b. }
\PY{l+s+sd}{    \PYZdq{}\PYZdq{}\PYZdq{}}
    \PY{n}{m}\PY{p}{,}\PY{n}{n} \PY{o}{=} \PY{n}{X}\PY{o}{.}\PY{n}{shape}
    \PY{n}{dj\PYZus{}dw} \PY{o}{=} \PY{n}{np}\PY{o}{.}\PY{n}{zeros}\PY{p}{(}\PY{p}{(}\PY{n}{n}\PY{p}{,}\PY{p}{)}\PY{p}{)}                            \PY{c+c1}{\PYZsh{}(n,)}
    \PY{n}{dj\PYZus{}db} \PY{o}{=} \PY{l+m+mf}{0.0}                                       \PY{c+c1}{\PYZsh{}scalar}

    \PY{k}{for} \PY{n}{i} \PY{o+ow}{in} \PY{n+nb}{range}\PY{p}{(}\PY{n}{m}\PY{p}{)}\PY{p}{:}
        \PY{n}{f\PYZus{}wb\PYZus{}i} \PY{o}{=} \PY{n}{sigmoid}\PY{p}{(}\PY{n}{np}\PY{o}{.}\PY{n}{dot}\PY{p}{(}\PY{n}{X}\PY{p}{[}\PY{n}{i}\PY{p}{]}\PY{p}{,}\PY{n}{w}\PY{p}{)} \PY{o}{+} \PY{n}{b}\PY{p}{)}          \PY{c+c1}{\PYZsh{}(n,)(n,)=scalar}
        \PY{n}{err\PYZus{}i}  \PY{o}{=} \PY{n}{f\PYZus{}wb\PYZus{}i}  \PY{o}{\PYZhy{}} \PY{n}{y}\PY{p}{[}\PY{n}{i}\PY{p}{]}                       \PY{c+c1}{\PYZsh{}scalar}
        \PY{k}{for} \PY{n}{j} \PY{o+ow}{in} \PY{n+nb}{range}\PY{p}{(}\PY{n}{n}\PY{p}{)}\PY{p}{:}
            \PY{n}{dj\PYZus{}dw}\PY{p}{[}\PY{n}{j}\PY{p}{]} \PY{o}{=} \PY{n}{dj\PYZus{}dw}\PY{p}{[}\PY{n}{j}\PY{p}{]} \PY{o}{+} \PY{n}{err\PYZus{}i} \PY{o}{*} \PY{n}{X}\PY{p}{[}\PY{n}{i}\PY{p}{,}\PY{n}{j}\PY{p}{]}      \PY{c+c1}{\PYZsh{}scalar}
        \PY{n}{dj\PYZus{}db} \PY{o}{=} \PY{n}{dj\PYZus{}db} \PY{o}{+} \PY{n}{err\PYZus{}i}
    \PY{n}{dj\PYZus{}dw} \PY{o}{=} \PY{n}{dj\PYZus{}dw}\PY{o}{/}\PY{n}{m}                                   \PY{c+c1}{\PYZsh{}(n,)}
    \PY{n}{dj\PYZus{}db} \PY{o}{=} \PY{n}{dj\PYZus{}db}\PY{o}{/}\PY{n}{m}                                   \PY{c+c1}{\PYZsh{}scalar}

    \PY{k}{for} \PY{n}{j} \PY{o+ow}{in} \PY{n+nb}{range}\PY{p}{(}\PY{n}{n}\PY{p}{)}\PY{p}{:}
        \PY{n}{dj\PYZus{}dw}\PY{p}{[}\PY{n}{j}\PY{p}{]} \PY{o}{=} \PY{n}{dj\PYZus{}dw}\PY{p}{[}\PY{n}{j}\PY{p}{]} \PY{o}{+} \PY{p}{(}\PY{n}{lambda\PYZus{}}\PY{o}{/}\PY{n}{m}\PY{p}{)} \PY{o}{*} \PY{n}{w}\PY{p}{[}\PY{n}{j}\PY{p}{]}

    \PY{k}{return} \PY{n}{dj\PYZus{}db}\PY{p}{,} \PY{n}{dj\PYZus{}dw}  
\end{Verbatim}
\end{tcolorbox}

    Run the cell below to see it in action.

    \begin{tcolorbox}[breakable, size=fbox, boxrule=1pt, pad at break*=1mm,colback=cellbackground, colframe=cellborder]
\prompt{In}{incolor}{ }{\boxspacing}
\begin{Verbatim}[commandchars=\\\{\}]
\PY{n}{np}\PY{o}{.}\PY{n}{random}\PY{o}{.}\PY{n}{seed}\PY{p}{(}\PY{l+m+mi}{1}\PY{p}{)}
\PY{n}{X\PYZus{}tmp} \PY{o}{=} \PY{n}{np}\PY{o}{.}\PY{n}{random}\PY{o}{.}\PY{n}{rand}\PY{p}{(}\PY{l+m+mi}{5}\PY{p}{,}\PY{l+m+mi}{3}\PY{p}{)}
\PY{n}{y\PYZus{}tmp} \PY{o}{=} \PY{n}{np}\PY{o}{.}\PY{n}{array}\PY{p}{(}\PY{p}{[}\PY{l+m+mi}{0}\PY{p}{,}\PY{l+m+mi}{1}\PY{p}{,}\PY{l+m+mi}{0}\PY{p}{,}\PY{l+m+mi}{1}\PY{p}{,}\PY{l+m+mi}{0}\PY{p}{]}\PY{p}{)}
\PY{n}{w\PYZus{}tmp} \PY{o}{=} \PY{n}{np}\PY{o}{.}\PY{n}{random}\PY{o}{.}\PY{n}{rand}\PY{p}{(}\PY{n}{X\PYZus{}tmp}\PY{o}{.}\PY{n}{shape}\PY{p}{[}\PY{l+m+mi}{1}\PY{p}{]}\PY{p}{)}
\PY{n}{b\PYZus{}tmp} \PY{o}{=} \PY{l+m+mf}{0.5}
\PY{n}{lambda\PYZus{}tmp} \PY{o}{=} \PY{l+m+mf}{0.7}
\PY{n}{dj\PYZus{}db\PYZus{}tmp}\PY{p}{,} \PY{n}{dj\PYZus{}dw\PYZus{}tmp} \PY{o}{=}  \PY{n}{compute\PYZus{}gradient\PYZus{}logistic\PYZus{}reg}\PY{p}{(}\PY{n}{X\PYZus{}tmp}\PY{p}{,} \PY{n}{y\PYZus{}tmp}\PY{p}{,} \PY{n}{w\PYZus{}tmp}\PY{p}{,} \PY{n}{b\PYZus{}tmp}\PY{p}{,} \PY{n}{lambda\PYZus{}tmp}\PY{p}{)}

\PY{n+nb}{print}\PY{p}{(}\PY{l+s+sa}{f}\PY{l+s+s2}{\PYZdq{}}\PY{l+s+s2}{dj\PYZus{}db: }\PY{l+s+si}{\PYZob{}}\PY{n}{dj\PYZus{}db\PYZus{}tmp}\PY{l+s+si}{\PYZcb{}}\PY{l+s+s2}{\PYZdq{}}\PY{p}{,} \PY{p}{)}
\PY{n+nb}{print}\PY{p}{(}\PY{l+s+sa}{f}\PY{l+s+s2}{\PYZdq{}}\PY{l+s+s2}{Regularized dj\PYZus{}dw:}\PY{l+s+se}{\PYZbs{}n}\PY{l+s+s2}{ }\PY{l+s+si}{\PYZob{}}\PY{n}{dj\PYZus{}dw\PYZus{}tmp}\PY{o}{.}\PY{n}{tolist}\PY{p}{(}\PY{p}{)}\PY{l+s+si}{\PYZcb{}}\PY{l+s+s2}{\PYZdq{}}\PY{p}{,} \PY{p}{)}
\end{Verbatim}
\end{tcolorbox}

    \textbf{Expected Output}

\begin{verbatim}
dj_db: 0.341798994972791
Regularized dj_dw:
 [0.17380012933994293, 0.32007507881566943, 0.10776313396851499]
\end{verbatim}

    \hypertarget{rerun-over-fitting-example}{%
\subsection{Rerun over-fitting
example}\label{rerun-over-fitting-example}}

    \begin{tcolorbox}[breakable, size=fbox, boxrule=1pt, pad at break*=1mm,colback=cellbackground, colframe=cellborder]
\prompt{In}{incolor}{ }{\boxspacing}
\begin{Verbatim}[commandchars=\\\{\}]
\PY{n}{plt}\PY{o}{.}\PY{n}{close}\PY{p}{(}\PY{l+s+s2}{\PYZdq{}}\PY{l+s+s2}{all}\PY{l+s+s2}{\PYZdq{}}\PY{p}{)}
\PY{n}{display}\PY{p}{(}\PY{n}{output}\PY{p}{)}
\PY{n}{ofit} \PY{o}{=} \PY{n}{overfit\PYZus{}example}\PY{p}{(}\PY{k+kc}{True}\PY{p}{)}
\end{Verbatim}
\end{tcolorbox}

    In the plot above, try out regularization on the previous example. In
particular: - Categorical (logistic regression) - set degree to 6,
lambda to 0 (no regularization), fit the data - now set lambda to 1
(increase regularization), fit the data, notice the difference. -
Regression (linear regression) - try the same procedure.

    \hypertarget{congratulations}{%
\subsection{Congratulations!}\label{congratulations}}

You have: - examples of cost and gradient routines with regularization
added for both linear and logistic regression - developed some intuition
on how regularization can reduce over-fitting

    \begin{tcolorbox}[breakable, size=fbox, boxrule=1pt, pad at break*=1mm,colback=cellbackground, colframe=cellborder]
\prompt{In}{incolor}{ }{\boxspacing}
\begin{Verbatim}[commandchars=\\\{\}]

\end{Verbatim}
\end{tcolorbox}


    % Add a bibliography block to the postdoc
    
    
    
\end{document}
